\subsubsection{本走行の結果}
きなこチームは本走行開始直後, グローバルパスに沿って進まず想定外の左旋回を行い道路に出たため, 走行距離6mで終了となった. 
このときのログを確認したところ, 正常なナビゲーションが行われていないことが分かった. 
大会当日, グローバルパス生成後に緊急停止スイッチを押し2分ほど待機すると, navigation2の更新周期が維持できなくなり, 処理遅延やタイムアウトが頻発するという問題が発生した. 
通常, このような問題はCPUやメモリの過負荷状態で観察される現象である. 
この問題は大会当日は再現性を持って発生したものの, 後日の検証では再現することができなかった. 
今後, 根本的な原因を特定するために, さらなるの調査を行っていく予定である. 

\subsubsection{実験走行で発生した問題}
実験走行中に以下の技術的な問題が発生した. 

1. 狭い通路で起きた問題\\
実験走行時, 狭い通路でロボットが回転し止まらなくなることが何度かあった. 
navigation2のログを解析した結果, 以下の処理が行われていたことが確認できた. 

\begin{enumerate}
  \item 障害物との距離が近いためロボットの周囲ローカルコストマップが飽和
  \item ローカルプランナーのパスを生成できずリカバリの動作として回転を開始
  \item  一定時間回転後に速度指令値を0にする処理を実行
\end{enumerate}

しかし, 3 の処理が実行されていたにも関わらず, 実際のロボットは静止せず回転を続けていた. 
実際の挙動とシステムの指令値の不一致について, 原因を究明するための調査を行っていく予定である. 

2. 動的障害物への対応の遅れ\\
グローバルパスに侵入する動的障害物に対するグローバルプランナーの更新が遅く, オペレーターが緊急停止スイッチを押さなければ衝突してしまう場面があった. 
また, システムが動的障害物の進行方向に経路を設定することがあり, 円滑な走行の妨げとなった. 
これに対する対策として, グローバルプランナーの更新周期の最適化と動的な障害物の移動方向を考慮したグローバルパスの生成のアルゴリズム改良が挙げられる. 