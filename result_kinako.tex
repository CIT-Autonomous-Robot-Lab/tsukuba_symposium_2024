\subsubsection{本走行の結果}
%#きなこチームは本走行開始直後、グローバル経路の通りに進まず想定外の左旋回を行い道路に出たため、走行距離6mで終了となった。
きなこチームは本走行開始直後に左旋回を行い、道路にコースアウトしてしまった。
%%%正常なナビゲーションが行われていなかったデータがほしい
%%% or 〜〜があった。そのため、正常なナビゲーションが行われていなかったと考えられる。
このとき、正常なナビゲーションが行われていなかった。
大会当日、グローバル経路生成後に緊急停止スイッチを操作し待機状態にすると、Nav2の更新周期が維持できなくなり、処理遅延やタイムアウトが頻発するという問題が発生した。
通常、このような問題はCPUやメモリの過負荷状態で観察される現象である。
この問題は大会当日は再現性を持って発生したものの、後日の検証では再現することができなかった。
今後、根本的な原因を特定するために、さらなる調査を行っていく予定である。

\subsubsection{実験走行で発生した問題}
実験走行中に以下の技術的な問題が発生した。

1. 狭所での不具合\\
実験走行時、狭所でロボットが回転し止まらなくなることが何度かあった。
Nav2のログ解析から、以下のプロセスが確認できた。

\begin{itemize}
  \item 障害物との距離が近いためロボットの周囲ローカルコストマップが飽和
  \item ローカルプランナーのパスを生成できずリカバリの動作として回転を開始
  \item  一定時間回転後に速度指令値を0にする処理を実行
\end{itemize}

しかし、実際のロボットは静止せず回転を続けていた。
実際の挙動とシステムの指令値の不一致について、原因究明のための追加調査を行っていく予定である。

2. 動的障害物への対応の遅れ\\
他にも動的障害物に対するグローバル経路の更新が遅く、衝突リスクが発生する場面があった。
また、システムが動的障害物の進行方向に経路を設定することがあり、円滑な走行の妨げとなった。
