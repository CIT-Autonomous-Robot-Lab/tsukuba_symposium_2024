\subsubsection{本走行の結果}
ツナチームの本走行の結果は、確認走行区間の最後の課題である障害物回避を失敗したため、
440[m]という結果になった。障害物回避の課題までは、ある一箇所を除いて、終始安定して走行できていた。
その一箇所は市役所の入口付近であり、点字ブロックの上を走行していたため
オドメトリに誤差が多く生じ、ロボットの位置と推定位置が縦方向にずれてしまった。
そのまま自己位置推定が破綻すると思われたが修正されたため、危機を脱した。
この状況は、動画\cite{ike_nav_loc_youtube}の(22:45〜24:45)に記録されている。

ツナチームは、7/15、7/16、11/4、11/8、11/9の計5日間、実験走行および本走行に参加した。
ike\_navは7月末から開発され、11/7に実機で使える状態になり、
動作確認ができたのは11/8、11/9のみであった。
%そのため、直前ギリギリまで勧めていた作業としては、まずまずの結果といえる。
一方、記録は表1のように440[m]となり、ike\_navは、
実機に搭載する前から不具合の少ないソフトウェアであったと言える。

\subsubsection{本走行・実験走行で見つかった課題}
\paragraph{障害物回避の不安定さ}
本走行時、ike\_navの障害物回避の機能にはバグが含まれていた。
A*によって障害物回避をするようなパスを生成し、それを追従することで障害物回避としていた。
このA*では、推定位置からゴールまでの最短経路を導出しつつ、障害物回避を行うために、
独自のヒューリスティック関数を実装した。
これが、うまく実装できていれば良かったが、距離が遠いほど障害物回避を回避しようとしなくなるような
実装となっていた。そのため今後は,障害物回避の安定のために,このヒューリスティック関数の実装を見直し,修正をする予定である.

% \paragraph{尤度場、コストマップの作成に時間がかかる}
% ike\_navでは、尤度場とコストマップを自律走行を開始させる前に作成する処理がある。
% 地図としては、確認走行区間のみの情報しかないが、
% この処理に、1分ほどかかってしまっていた。
% これは、尤度場、コストマップの作成の実装に問題があるため、
% 作成部分の実装を見直し、修正をする予定である。
% @@@1分くらいいいんじゃない?@@@