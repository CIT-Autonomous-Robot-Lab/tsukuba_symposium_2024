\subsubsection{本走行の結果}
きなこチームは本走行開始直後, グローバル経路に沿って進まず想定外の左旋回を行い道路に出たため, 走行距離6mで終了となった. 
このときのログを確認したところ, 正常なナビゲーションが行われていないことが分かった. 
%%%正常なナビゲーションが行われていなかったデータがほしい
%%% or 〜〜があった. そのため, 正常なナビゲーションが行われていなかったと考えられる. 
大会当日, グローバル経路生成後に緊急停止スイッチを押し2分ほど待機すると, navigation2の更新周期が維持できなくなり, 処理遅延やタイムアウトが頻発するという問題が発生した. 
%通常, このような問題はCPUやメモリの過負荷状態で観察される現象である. 
%@@@↑たんなる対策漏れです(バグの一種)
%この問題は大会当日は再現性を持って発生したものの, 後日の検証では再現することができなかった. 
%今後, 根本的な原因を特定するために, さらなる調査を行っていく予定である. 
%@@@↑こんなの結局何の対策もできない人の言うことです. 

この不具合の原因については調査中であるが, 
根本的な原因は, システムの不安定さを残したまま
開発に走ってしまったことにある. 
「つくばチャレンジでスタートに失敗する」という事例については, 
チームメンバーの上田が文献\cite{上田確率}において確率や統計に基づいて考察し, 
度々対策をメンバーに伝えていた. しかし, 
メンバーが他のことに気を取られて優先度を下げてしまい, 
結果的に忠告通りの結果になってしまった. 


\subsubsection{実験走行で発生した問題}
実験走行においては, 以下の技術的な問題が発生した. 

1. 隘路で起きた問題\\
実験走行時, 狭い通路でロボットが回転し止まらなくなることが何度かあった. 
navigation2のログを解析した結果, 以下の処理が行われていたことが確認できた. 

\begin{enumerate}
  \item 障害物との距離が近いためロボットの周囲ローカルコストマップが飽和
  \item ローカルプランナーのパスを生成できずリカバリの動作として回転を開始
  \item  一定時間回転後に速度指令値を0にする処理を実行
\end{enumerate}

しかし, 3の処理が実行されていたにも関わらず, 実際のロボットは静止せず回転を続けていた. 
実際の挙動とシステムの指令値の不一致について, 原因を究明するための調査を行っていく予定である. 

2. 動的障害物への対応の遅れ\\
動的障害物に対するグローバル経路の更新が遅く, オペレーターが緊急停止を押さなければ衝突してしまう場面があった. 
また, システムが動的障害物の進行方向に経路を設定することがあり, 円滑な走行の妨げとなった. 
これに対する対策として, グローバルプランナーの更新周期の最適化と動的な障害物の移動方向を考慮したグローバル経路の生成のアルゴリズム改良が挙げられる. 
