\subsubsection{本走行の結果}
きなこチームは本走行のスタート直後左に旋回し、道路に出たため走行を終了した。
走行距離は6mとなった。
このとき、正常なナビゲーションが行われていなかった。
大会当日、グローバル経路を生成後、緊急停止スイッチを押してしばらく待つとバグが発生した。
発生した事象は、Nav2が予め設定された更新周期を維持できず、処理遅延やタイムアウトが頻発するというものだった。
通常、このような問題はCPUやメモリの負荷が高い場合に発生する。

大会当日はこの問題が常に再現できたが、後日のテストでは同様の問題を再現することができなかった。
この問題の根本的な原因を明らかにするために、追加の調査を行っていく予定である。

\subsubsection{実験走行で発生した問題・課題}
狭い箇所で車体が回転し止まらなくなることがあった。
そのときのログが残っているので解析すると、次のことが分かった。

\begin{itemize}
  \item 障害物が近いためロボットの周囲がローカルコストで埋まる
  \item ローカルプランナーのパスがどこにも生成できずリカバリの回転動作実行
  \item  一定時間回転後に速度指令値を0にする処理が実行される。
\end{itemize}
しかし、実際のロボットは止まっていなかった。


% 動的障害物グローバル経路の切り替えが遅く衝突しかける。


% 回転動作でのタイムアウト時のコードをみる。
% ▶BTは回転がタイムアウトして止める司令を出している。しかし、実際は回転し続けた。速度指令値がどうなっているか?