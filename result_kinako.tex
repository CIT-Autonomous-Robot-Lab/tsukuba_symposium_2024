\subsubsection{本走行・実験走行で見つかった課題}
\paragraph{ロボットの旋回による自己位置の破綻}


たまチームはつくばチャレンジEXでの屋内での自律走行を実施した。
その際、IMUを搭載していなかったため、
ナビゲーション中にロボットが人混みを避けようとして頻繁に旋回し、
自己位置が破綻するという課題が生じた。
この課題への対策としてIMUを実装し、
速度指令値から求めた角速度を積分した値であるヨー軸の角度を、
IMUから得られるヨー方向の角速度の積分値に置き換えた。
その結果、
ナビゲーション中には何度も障害物を回避しようとして旋回することがあったが、
これらの旋回が自己位置推定の破綻には繋がらなかったという結果が得られた。


IMUの搭載による効果は、
マッピングの際にも確認できた。
マップの取得では、
人がコントローラを用いてロボットを操作した。
IMUを搭載していなかった際には、
操作ミスやタイヤのスリップによる不要な旋回が原因で、
歪んだマップが生成されることがあった。
しかし、IMUの搭載によりマップの歪みが減少した。
% マップの比較画像 %


\paragraph{マップの歪みが解消しきれていない課題}


IMUを搭載してマッピングを行ったが、
ループクローズが行われず、
結局歪んだマップしか取れなかった。
この問題に対処するため、
最終的にペイントツールを使用して人力で歪みを修正した。
しかし、これは根本的な解決ではなく、
問題の原因を究明し、
本質的な解決策を見つける必要がある。
