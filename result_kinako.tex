\subsubsection{本走行の結果}
きなこチームは本走行のスタート直後左に旋回し、道路に出たため走行を終了した。
走行距離は6mとなった。
このとき、正常なナビゲーションが行われていなかった。
大会当日、モーターに動作許可を与えるコマンドを2度実行すると、システムにバグが発生した。
発生した事象は、Nav2が予め設定された更新周期を維持できず、処理遅延やタイムアウトが頻発した。
通常、このような問題はCPUやメモリの負荷が高い場合に発生する。
しかし、モーターへの動作許可コマンドはシステムに影響を与えるものではなく、直接的な原因とは考えられない。
大会当日はこの問題が常に再現できたが、後日のテストでは同様の問題を再現することができなかった。
この問題の根本的な原因を明らかにするために、追加の調査を行っていく予定である。

\subsubsection{実験走行で発生した問題・課題}
狭い箇所で車体が回転し始めることがなんどかあった。
これは壁が近いせいでローカルプランナーのパスがどこにも生成できずリカバリの回転動作を行うが、四方がローカルコストで埋まり、どこにも進めずひたすら回転をしていたのではないかと考えられる。
ロボットや環境に合わせた適切なチューニングが必要であると考えられる。
動的障害物グローバル経路の切り替えが遅く衝突しかける。


自己位置推定の破綻が最終的な問題かも?
回転動作でのタイムアウト時のコードをみる。